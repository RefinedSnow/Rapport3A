% !TeX encoding = UTF-8
% !TeX spellcheck = fr_FR
\documentclass[french,12pt,a4paper,titlepage,openright,openbib]{report}

\usepackage[utf8]{inputenc}
\usepackage[T1]{fontenc}
\usepackage[english,french]{babel}
\usepackage[default]{gillius}

\usepackage{lipsum}
\usepackage{array}
\usepackage[pdfborder={0 0 0}]{hyperref}
\usepackage[xindy,toc]{glossaries}
\usepackage{titlesec}
\usepackage{color}
\usepackage{blindtext}
\usepackage{fancyhdr}
\usepackage{lastpage}

\pagestyle{fancy}
\fancyhf{}

\fancyfoot[C]{Page \textbf{\thepage} sur \textbf{\pageref{LastPage}}}

\fancypagestyle{plain}{%
	\fancyhf{} % clear all header and footer fields
	\fancyfoot[C]{Page \textbf{\thepage} sur \textbf{\pageref{LastPage}}} % except the center
	\renewcommand{\headrulewidth}{0pt}
	\renewcommand{\footrulewidth}{0pt}}


\renewcommand{\headrulewidth}{0pt}
\renewcommand{\footrulewidth}{0pt}

\definecolor{gray75}{gray}{0.75}
\newcommand{\hsp}{\hspace{20pt}}

\parindent=0in
\parskip=8pt
\renewcommand\thechapter{\Roman{chapter}}
\renewcommand\thesection{\Roman{section}}
\titleformat{\chapter}[hang]{\Huge\bfseries}{\thechapter\hsp\textcolor{gray75}{|}\hsp}{0pt}{\Huge\bfseries}

\makeglossaries

\newglossaryentry{mifare}
{
	name=MIFARE,
	description={Marque de NXP incluant une large gamme de circuits intégrés sans contact.},
}
\newacronym{nfc}{NFC}{Near Field Communication}

\begin{document}

\title{Rapport de mi-parcours de troisième année}
\author{Jennifer Gr\"{a}nicher}
\date{12 Février 2018}
\maketitle

\chapter*{Remerciements}

J'aimerais remercier toutes les personnes qui ont contribué au bon déroulement de ces trois dernières années d'apprentissage, au sein de l'entreprise NXP Semiconductors, et au sein de l'ENSICAEN.

Je tiens particulièrement à remercier Virginie Jobard qui m'a épaulée et aidée à prendre mes repères au début.

Je souhaiterais également remercier Didier Graignic qui a su prendre le relais et m'aiguiller au cours de ma mission sur le projet MOOCTab.

J'aimerais aussi remercier Nicolas Guillerm pour toute l'aide qu'il a apporté que ce soit pour le rapport ou les missions au sein de l'entreprise.

Je remercie également Wilfried Aubry qui a été de bon conseil et qui a su apporter une grande aide.

Je tiens également à remercier mes collègues chez NXP pour leur aide et leurs soutien.


\tableofcontents

\chapter*{Historique}
\begin{table}[ht]
	\label{tab:historique}
	\centering
	\begin{tabular}{|c|c|c|c|}
		\hline
		{\bf Version} & {\bf Date} & {\bf Rédigé par}    & {\bf Mise à jour}    \\
		\hline
		1.0           & 29/01/2018 & Jennifer Gränicher  & Création du document \\
		\hline
		1.1           & 06/02/2018 & Jennifer Gränicher  & Mise en page \\
		\hline
	\end{tabular}
\end{table}

{\let \clearpage \relax \chapter*{Diffusion}}
\begin{table}[ht]
	\label{tab:diffusion}
	\centering
	\begin{tabular}{|c|c|c|c|}
		\hline
		{\bf Destinataire} & {\bf Société}      & {\bf Fonction}   		 & {\bf Date}\\
		\hline
		Nicolas Guillerm   & NXP Semiconductors & Maitre d'apprentissage & 08/02/2018 \\
		\hline
		Wifried Aubry      & ENSICAEN 			& Tuteur				 & 12/02/2018 \\
		\hline
	\end{tabular}
\end{table}

\chapter{Introduction}

Ce document parle de mes trois années d'apprentissage au sein de NXP Semiconductors, de Septembre 2015 à Janvier 2018.

Une première partie concerne les derniers changements au sein de l'entreprise.
Une seconde partie concerne les projets en cours, elle est centrée sur la dernière année.
Enfin, une autre partie traite de mon évolution personnelle et professionnelle au cours de ces trois années.

\chapter{Actualités}

Au cours des derniers mois les effectifs de NXP à Caen ont été réduits. Cela a eu pour conséquence un réaménagement de certains bureaux.
Didier Graignic qui remplaçait Virginie Jobard a quitté l'entreprise, lui aussi, Nicolas Guillerme l'a remplacé en qualité de maître d'apprentissage.
L'équipe avec laquelle je travaille est aujourd'hui composée de Nicolas Guillerme, Dominique Defossez et moi même. Cette équipe est l'équipe attachée au projet MOOCTab que je détaillerai plus loin.


\section{Qualcomm}

En octobre 2016 Qualcomm annonçait son souhait de racheter NXP Semiconductors. La commission américaine avait approuvé ce rachat mais la commisions européenne n'était pas d'accord. Après une enquête approfondie, la commission européenne accepte le rachat uniquement sous certaines conditions :
Qualcomm n'acquérira pas les brevets essentiels au standard \gls{nfc} de NXP, ni quelques uns non-essentiels concernant la \gls{nfc}.
Qualcomm devra maintenir, pendant une période de huit ans, la licence actuelle de la technologie \gls{mifare}. Enfin ils devront aussi garantir l'intéropérabilité des puces NXP avec les autres constructeurs.

\chapter{Missions}
\lipsum[1-1]
\section{MOOCTab}
\lipsum[2-3]
\section{NFC Playbook}
\lipsum[4-5]
\chapter{Évolution}
Lorsque je suis arrivée chez NXP la première fois je n'avais aucune expérience en entreprise. J'ai commencé par un stage de douze semaines, ce stage m'avait permis de faire mes premiers pas dans le monde professionnel et de trouver mon poste d'apprentie.
Cela fait maintenant deux ans et demi que j'évolue au fil des mois autant en entreprise qu'à l'école.
\section{Technique}
\lipsum[1-1]
\section{Collaboration}
\lipsum[1-1]
\chapter{Bilan}
\lipsum[6-7]
\chapter{Conclusion}
\lipsum[8-9]

\printglossary[title={Glossaire}]

\end{document}


