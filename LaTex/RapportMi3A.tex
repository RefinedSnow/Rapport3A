% !TeX encoding = UTF-8
% !TeX spellcheck = fr_FR
\documentclass[french,12pt,a4paper,titlepage,openright,openbib]{report}

\usepackage[utf8]{inputenc}
\usepackage[T1]{fontenc}
\usepackage[english,french]{babel}
\usepackage{helvet}

\usepackage{array}
\usepackage{acronym}
\usepackage{titlesec}
\usepackage{color}
\usepackage{blindtext}

\definecolor{gray75}{gray}{0.75}
\newcommand{\hsp}{\hspace{20pt}}

\parindent=0in
\parskip=8pt
\renewcommand\thechapter{\Roman{chapter}}
\titleformat{\chapter}[hang]{\Huge\bfseries}{\thechapter\hsp\textcolor{gray75}{|}\hsp}{0pt}{\Huge\bfseries}

\begin{document}

\title{Rapport de mi-parcours de troisième année}
\author{Jennifer Gr\"{a}nicher}
\date{12 Février 2018}
\maketitle

\chapter*{Remerciements}

J'aimerais remercier toutes les personnes qui ont contribué au bon déroulement de ces trois dernières années d'apprentissage, au sein de l'entreprise NXP Semiconductors, et au sein de l'ENSICAEN.

Je tiens particulièrement à remercier Virginie Jobard qui m'a épaulée et aidée à prendre mes repères au début.

Je souhaiterais également remercier Didier Graignic qui a su prendre le relais et m'aiguiller au cours de ma mission sur le projet MOOCTab.

J'aimerais aussi remercier Nicolas Guillerm pour toute l'aide qu'il a apporté que ce soit pour le rapport ou les missions au sein de l'entreprise.

Je remercie également Wilfried Aubry qui a été de bon conseil et qui a su apporter une grande aide.

Je tiens également à remercier mes collègues chez NXP pour leur aide et leurs soutien.


\tableofcontents

\chapter*{Historique}
\begin{table}[ht]
	\label{tab:historique}
	\centering
	\begin{tabular}{|c|c|c|c|}
		\hline
		{\bf Version} & {\bf Date} & {\bf Rédigé par}    & {\bf Mise à jour}    \\
		\hline
		1.0           & 29/01/2018 & Jennifer Gränicher  & Création du document \\
		\hline
		1.1           & 06/02/2018 & Jennifer Gränicher  & Mise en page \\
		\hline
	\end{tabular}
\end{table}

{\let \clearpage \relax \chapter*{Diffusion}}
\begin{table}[ht]
	\label{tab:diffusion}
	\centering
	\begin{tabular}{|c|c|c|c|}
		\hline
		{\bf Destinataire} & {\bf Société}      & {\bf Fonction}   		 & {\bf Date}\\
		\hline
		Nicolas Guillerm   & NXP Semiconductors & Maitre d'apprentissage & 08/02/2018 \\
		\hline
		Wifried Aubry      & ENSICAEN 			& Tuteur				 & 12/02/2018 \\
		\hline
	\end{tabular}
\end{table}

\chapter{Introduction}

\end{document}


